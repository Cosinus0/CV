\documentclass[10pt,a4paper]{letter}
\usepackage[utf8x]{inputenc}
\usepackage{ucs}
\usepackage{amsmath}
\usepackage{amsfonts}
\usepackage{amssymb}
\usepackage{graphicx}
\usepackage{hyperref}
\usepackage[danish]{babel}
\signature{Nathan Hugh Barr}
\address{Nathan Hugh Barr \\ Østerbrogade 202 4.tv \\ 2100 København \\ Danmark}
\begin{document}

\begin{letter}{}
\opening{\textbf{Ansøgning: Gymnasielærer i Matematik}}

Jeg er Cand.Scient. i Fysik og Matematik som brænder for at vise de sjove og spændende hjørne i de to fag, og hvordan matematik kan bruges i den virkelige verden. 

For mig, det sjovt i at undervise matematik er formidling af kompleks koncepter og objekter så at de bliver håndgribelige koncepter. Jeg synes også en sjov del af undervisning i matematik er at forstå hvad de elever har svært ved og guider dem mod forståelsen.

Jeg har fået min fagligheder fra både min uddannelse og de arbejde jeg har haft ved siden af. Jeg har været en tutor for gymnasium elever i både Fysik og Matematik. Jeg har arbejdet med dem en mod en og har sat mig selv i deres svagheder og har planlægt et forløb som har styrket deres kompetencer i de to fag. Jeg har også været en del af RUC scienceshow, hvor vi har udviklet en show på 45 minutter som har afspejlet de forskellige grene i naturvidenskab og nogen naturvidenskabelig projekter lavede på RUC. Jeg har fået en del træning i formidling og hvordan man underholder en undervisning sal fuldt af gymnasium elever. 

Min modersmål er engelsk så jeg både kan kommunikere på dansk og engelsk. Jeg er en optimistisk og rar person som leder efter udfordringer.  

Jeg glæder mig til at høre fra jer om jobbet.  

\closing{Dbh.}

\end{letter}

%I feel that I am fit for the position due to the following points. Physics as a discipline is based on a strong theoretical fundament and the ability to solve problems and I have this competence. I have an analytical competence through studying mathematics.
\end{document}