\documentclass[10pt,a4paper]{letter}
\usepackage[utf8x]{inputenc}
\usepackage{ucs}
\usepackage{amsmath}
\usepackage{amsfonts}
\usepackage{amssymb}
\usepackage{graphicx}
\usepackage{hyperref}
\usepackage[margin=0.5in]{geometry}
\signature{Nathan Hugh Barr}
\address{Nathan Hugh Barr \\ Østerbrogade 202 4.tv \\ 2100 København Ø \\ Danmark}
\begin{document}

\begin{letter}{}
\opening{Til rette vedkommende:}
Jeg er en nyuddannet kandidat i fysik og matematik som leder efter en virksomhedspraktik indenfor "data analytics". Igennem min uddannelse har jeg fået en del kompetencer indenfor dataanalyse, matematisk modellering, beregningsfysik og tværfagligt samarbejde, som jeg er sikker på kan gavne din virksomhed.

\underline{\textbf{Projektarbejde}}

Jeg har erfaring med problemorienterede projekter, som har været tværfaglige indenfor naturvidenskab. Jeg kan arbejde selvstændig og også i en gruppe. Når jeg arbejder på et projekt, er jeg selvdrevet, og jeg har planlægningskompetencer, som gør mig i stand til at nå deadlines.   

 
\underline{\textbf{Problem solver}}

Jeg har erfaring med at være en "Problem solver", når jeg arbejder på projekter. Jeg har erfaring med at koge problemer ned til kernen og bearbejder det, uden at tabe overblikket. Jeg belyser problemer med mine numeriske og analytiske kompetencer, som jeg har oparbejdet gennem min uddannelse, og jeg når frem til den bedste løsning af problemet.
   

\underline{\textbf{Komplet dataanalytiker}}

Jeg har kendskab til programmeringssprogene Matlab og Python, hvor jeg kan udvinde og analysere data fra databaser. Jeg har kendskab til scripting, hvor jeg kan automatisere analyseprocessen, og jeg kan implementere matematiske modeller til at analyse tendenser i data. Jeg har også kendskab til forskellige måder at visualisere data på og deres fordele og ulemper. Jeg kan drage konklusioner fra data og præsentere de konklusioner til et bredt publikum. 


\underline{\textbf{Tværfaglig uddannelse}}

Fra min tværfaglige uddannelse i fysik og matematik har jeg fået en analytisk og struktureret tilgang, når jeg arbejde. Jeg har en kritisk sans, når jeg arbejder med data, og jeg har god dataetik. Jeg har et stærk fundament i matematik og fysik, som gør mig i stand til at tackle abstrakt problemstillinger. Min tværfaglighed gør samtidig, at jeg har stiftet bekendtskab med andre fagområder, og jeg kan derfor sagtens omstille mig til at arbejde med forskelligartede emner og områder. Om det måtte være indenfor demografi, økonomi eller biologi.  
\\

Her til sidst vil jeg give et overblik over, hvordan jeg er som medarbejder. Jeg er en ambitiøs og optimistisk person, som er klar til at hjælpe når som helst med hvad som helst. Jeg er lærenem, effektiv når jeg arbejder på opgaven, og jeg er ikke bange for udfordringer og ansvar. Jeg vil meget gerne mødes til et kaffemøde, hvor vi kan snakke om jeres nuværende projekter, og hvordan jeg kan hjælpe.  
   
\closing{Med venlig hilsen,}

\end{letter}
\end{document}